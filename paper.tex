
\documentclass{article}

\usepackage{graphicx}
\usepackage{hyperref}
\usepackage{amsmath}
\usepackage{algorithm}
%\usepackage{algorithmic}
\usepackage{algpseudocode}
\usepackage{subcaption}
\usepackage{booktabs}


\usepackage[
  paper  = letterpaper,
  left   = 1.25in,
  right  = 1.25in,
  top    = 1.0in,
  bottom = 1.0in,
  ]{geometry}
\usepackage{setspace}

% For importing python source code, from https://tex.stackexchange.com/questions/83882/how-to-highlight-python-syntax-in-latex-listings-lstinputlistings-command

\usepackage{color}
\definecolor{deepblue}{rgb}{0,0,0.5}
\definecolor{deepred}{rgb}{0.6,0,0}
\definecolor{deepgreen}{rgb}{0,0.5,0}

% Default fixed font does not support bold face
\DeclareFixedFont{\ttb}{T1}{txtt}{bx}{n}{12} % for bold
\DeclareFixedFont{\ttm}{T1}{txtt}{m}{n}{12}  % for normal
\DeclareFixedFont{\ttfuck}{T1}{txtt}{m}{n}{12}  % for normal


\usepackage{listings}

% Python style for highlighting
\newcommand\pythonstyle{\lstset{
language=Python,
basicstyle=\ttm,
otherkeywords={self},             % Add keywords here
keywordstyle=\ttb\color{deepblue},
emph={MyClass,__init__},          % Custom highlighting
emphstyle=\ttb\color{deepred},    % Custom highlighting style
stringstyle=\color{deepgreen},
frame=tb,                         % Any extra options here
showstringspaces=false,             
breaklines=true,                 % come from https://tex.stackexchange.com/questions/243500/python-code-block-outside-of-margin
}}


% Python environment
\lstnewenvironment{python}[1][]
{
\pythonstyle
\lstset{#1}
}
{}

% Python for external files
\newcommand\pythonexternal[2][]{{
\pythonstyle
\lstinputlisting[#1]{#2}}}

% Python for inline
\newcommand\pythoninline[1]{{\pythonstyle\lstinline!#1!}}



\graphicspath{{images/}}

\title{Hidden enemy detecting using bayes inference}

\author{yueyi zhuo(2014060843) 
\footnote{The LaTeX source code and experiment files is hosted on GitHub, see: 
\url{https://github.com/yiyuezhuo/Undergraduate-thesis}}}

\date{2018}

\begin{document}

\maketitle

\begin{abstract}

Assuming a location which have samller distance to two conflict factions have higher probability causing a battle,
how do we infer the location of enemy if we can only observe the locations of allies and battles? 
The latent variable inference problem requieres a suitable front probability model and a effiecent way to run the 
inference. In this work, a battle model is introduced and a bayes inference framework based on pytorch 
is developed to reach the inference task including point, movement detection and 
exist probability estimation with different prior and various initialization value setting.
A case study about The battle of Gettysburg is included to show the overview of the content covered by the paper.

\end{abstract}

\tableofcontents



\section{Introduction}

In real warfare, you can't observe enemy units even allies units as RTS or classic board wargame way.
The only thing you can rely on are complex and confusing even self-contradiction information and misinformation.
There're a lot of example, that general misunderstand information and may be defeated due to this. 
Bing Sun manipulated field kitchen of his army to trick his opponent, General McClellan reject to attack
due to "peace gun"(fake gun instrument) seted by south rebels, lossing oppotunity to beat rebel down.

Unfortunately, current military decision similation program, 
such as "classic" wargame (a example see Figure~\ref{fig:hps}) and somewhat  "serious" RTS, 
tend to capture more "macro" aspect and simplify the reconnaissance and 
information modeling aspect into a unaccepted level.

\begin{figure}[h]
\includegraphics[width=0.6\linewidth]{SmolenskR4.jpg}
\caption{This figure show the reconnassance treatment of Smolensk' 41, 
a classic hex wargame developed by HPS.
Units can only be observed wholly or can't be observed. }
\label{fig:hps}
\end{figure}

Though we can read the oob(order of battle) as Figure~\ref{fig:metz} from map of battle, 
but it's obvious that those marks are placed in there using post Liang Zhuge information. 
In real time battle field, you can only hear where be attacked and where observe the trace of enemy 
(instead of unit detail id or somewhat weird strange quatity information), 
and some of them may be wrong and some of them may be released by enemy to confuse you.
The movement of Jackson in valley campaign is a vivid example about how the fake information intentionally
made by oppnent get the change to weaken the strength of army.

\begin{figure}[h]
\includegraphics[width=0.6\linewidth]{metz.jpg}
\caption{The map of battle of Metz, showing oob that can't be observed in real time battlefied. }
\label{fig:metz}
\end{figure}


There are some work were made to solve this problem 
\cite{hostetler2012inferring} \cite{vsmejkal2016integrating} \cite{touhou}.
Since they filter small part of information from replay file, they made a case that enemy can not be 
seen in that direct way, then they proposed according probability model to rebuild real situation.
Those're good tries, but not real demands for most of case. The paper we
\footnote{Though the paper only have a writer, 
but "we" seems a traditional and kind pronoun for such paper. 
In following content, both "we" and "I" will be used to represent the writer.} 
wrote will deal the problem in a more direct and intuitive way. 
A coordinate based position and movement detecting method is present in this paper.

\section{Building a proper forward model}


In forward model, a process that can give probility of number of battles and the mass in support space 
is required. For example, given ally and ememy location setting as Figure~\ref{fig:stateNoBattle}.

\begin{figure}[h]
\includegraphics{state_no_battle.png}
\caption{Ally and enemy setting}
\label{fig:stateNoBattle}
\end{figure}

\subsection{A basic setting}

Suppose the the number of battle is $10$ 
\footnote{ Though the number can modeled as it draw from a given distribution by adding 
stochastic process, it be not adpoted due to its extra difficulty in express.}
, we can model the $10$ battles are indepently drawn from below distribution:

$$
P(x,y) \propto \max_i p^A_{i}(x,y) \max_j p^E_{j} (x,y)
$$

Where $p^A_{i}(x,y)$ indicate the probability density "released" from of i’st ally unit,
$p^E_{j}$ indicate same thing from j’st enemy unit. And the density is defined by:

$$
P^S_i(x,y) \propto \exp(-\frac{1}{2}((x-x^S_i)^2 + (y-y^S_i)^2))
$$

Where the probability in $(x,y)$ released from i'th unit of S side($S \in \{A,E \}$). 
So the above formula can be rewritten as: 

$$
P(x,y) \propto \exp(-\frac{1}{2}((x-x_{nearst(x,y)})^2 + (y-y_{nearst(x,y)})^2))
$$

Where $nearst(x,y)$ point to the unit(enemy or ally) which is nearest to (x,y).

This means the non-normalized probability is only subject to the nearest two conflict units’s distance. 
Though the assumption don’t have abundant meaning,
it capture something crucial but have obvious weakness such as independent hypothesis, 
lately the model will be adujusted mutiple times to resolve this and other problems. 
Anyway, the probability can be computed and visualized as contour Figure~\ref{fig:stateNoBattleProb}.

\begin{figure}[h]
\includegraphics{state_no_battle_prob.png}
\caption{Battle density distribution}
\label{fig:stateNoBattleProb}
\end{figure}

The probability distribution don't have a standard shape, implying it's hard to sample. 
To overcome this, the "grid sampling" technology is used. In this example,
 the support of density is limited to $[-1,5] \times [0,6]$
(that is, the point out of set $[-1,5] \times [0,6]$ have zero density). 
And $100 \times 100$ points are selected to build a grid approxmation. 
See a size-reduced example serve to eye in Figure~\ref{fig:gridify}.

\begin{figure}[h]
\includegraphics[width=0.6\linewidth]{gridify.png}
\caption{$25 \times 25$ discrete approxmation}
\label{fig:gridify}
\end{figure}

So firstly we draw a square in the grid according to the approximated probability, 
and then draw two uniform variable $dX \sim U(0,5-(-1)/100),dY \sim U(0,(6-0)/100)$ 
to determine the position drew by this process.

Repeating the process 10 times, 
we can get size 10 independent sample from the distribution as such Figure~\ref{fig:stateSampleBattle}.

\begin{figure}[h]
\includegraphics[width=0.6\linewidth]{state_sample_battle.png}
\caption{Ally,enemy,prob, and a sample of battle}
\label{fig:stateSampleBattle}
\end{figure}

If the positions of enemy is hidden, 
then we can get a instance of problem data as Figure~\ref{fig:stateNoEnemy}.

\begin{figure}[h]
\includegraphics[width=0.6\linewidth]{state_no_enemy.png}
\caption{hide enemy}
\label{fig:stateNoEnemy}
\end{figure}

\subsection{A complex setting}

Although the distribution may satisfy our goal, 
but the $\min$ function will make that gradient-based optmization method difficult to effect 
in those units whose is not nearest to observed battle. 
So a another specification is proposed below to get it done:

First, a differentiable classifier is used to determine the "conflict level" in every point.
Supposing a point is more likely be controled by S side 
if a unit of S side have shortest distance to it, then we can define that the more two control probability 
being close to eachother, the more conflict chance in this point. So the conflict factor is defined by:

$$
P_{\text{conflict}}(x,y) = P_\text{ally}(x,y) P_\text{enemy}(x,y) = P_\text{ally}(x,y)(1-P_\text{ally}(x,y))
$$

Where $P_\text{ally}(x,y)$ indicate the probability(belief) of $(x,y)$ being classified 
into ally set given by classifer.

In this paper, we use naive bayes classifer as classifer since it helps gradient computation. 
The classification probability are given by:

$$
P_\text{ally}(x,y) = \frac{
N(x\mid \mu^A_X ,\sigma^A_X) N(y \mid \mu^A_Y, \sigma^A_Y)
}{
N(x \mid \mu^A_X , \sigma^A_X) N(y \mid \mu^A_Y , \sigma^A_Y) + 
N(x \mid \mu^E_X , \sigma^E_X N(y \mid \mu^E_Y , \sigma^E_Y)
}
$$

Where $N(x \mid \mu,\sigma)$ is normal probability density function:
$$
N(x \mid \mu,\sigma) = \frac{1}{\sqrt{2\pi \sigma^2}} \exp\left(\frac{(x-\mu)^2}{2\sigma^2}\right)
$$

The classifier parameters $\mu,\sigma$ are estimated by normal moment estimation. 

\begin{align*}
\mu_Z^S    &= 1/N^S \sum_{i=1}^N Z_i^S \\
\sigma_Z^S &= 1/N^S (Z_i^S - \mu_Z^S)^2
\end{align*}

Where $Z \in \{ X,Y \}$ and $S \in \{ A,E \}$. $N_S$ is number of units of side $S$.

A living example of classify probability can be seen in Figure~\ref{fig:naivebayes}.

\begin{figure}[h]
\includegraphics[width=0.6\linewidth]{naivebayes.png}
\caption{naive bayes predict probability coutour map}
\label{fig:naivebayes}
\end{figure}

Then, we define that the less the distance between point and any units(ally or enemy) be,
the more likely the point will occur a battle.

$$
P_{\text{distance}}(x,y) = \exp(-\alpha \min_{u} distance)
$$

Where $u$ means any unit. $\min_u distance$ the shortest distance in all possible distance 
between coordinate of unit and $(x,y)$.

So the probability is:

$$
P(x,y) = P_{\text{conflict}}(x,y) P_{\text{distance}}(x,y) = 
P_\text{ally}(x,y) P_\text{enemy}(x,y) \exp(-\alpha \min_{u} distance)
$$

Set $\alpha=0.1$, the 2d function can be seen in Figure~\ref{fig:combOne}:

\begin{figure}[h]
\includegraphics[width=0.6\linewidth]{comb1.png}
\caption{Combining a naive bayes classifier and a smooth distance factor}
\label{fig:combOne}
\end{figure}

The smooth probability seems good, but the sampled result seems too diffused 
(see Figure~\ref{fig:combTwo}) and if scale is enlarged, 
the problem even worsen (see Figure~\ref{fig:combThree}).
Some parameters tuning are tried but not satisfy the demand. 
So the smooth function family are removed and uniform distribution is employed. 

\begin{figure}[h]
\includegraphics[width=0.6\linewidth]{comb2.png}
\caption{Too diffused sample}
\label{fig:combTwo}
\end{figure}

\begin{figure}[h]
\includegraphics[width=0.6\linewidth]{comb4.png}
\caption{When extend the scale, unrational sample will present inevitably.}
\label{fig:combThree}
\end{figure}

\subsection{Empolyed setting}

We assign a point constant probability 
if it is greater than conflict threshold and less than distance threshold:

$$
P(x,y) \propto
\begin{cases}
1 & P_\text{ally}(x,y) (1-P_\text{ally}(x,y)) > \text{ conflict threshold and }
    \min_{u} distance < \text{conflict threshold} \\
0 & \text{otherwise}
\end{cases}
$$

Figure~\ref{fig:combFive} shows how the unrational sample point is suppressed by the way.

\begin{figure}[h]
\includegraphics[width=0.6\linewidth]{comb5.png}
\caption{Forced 0 zero probability can suppress the unrational point}
\label{fig:combFive}
\end{figure}

To get gradient information to proceed optimization, 
rewrite the conditional jump function to "soft" version, 
just like relation of ReLu and softplus in machine learning. 
For example, sigmmoid function $sigm(x) = \frac{1}{1+\exp(-x)}$:

$$
P(x,y) \propto sigm(t (P_\text{ally}(x,y) (1-P_\text{ally}(x,y)) - \theta_0)) sigm(t(-\min_{u} distance + \theta_1))
$$

Where $t$ is "tense" parameter controling the shape of sigmoid, $\theta_0,\theta_1$ are threshold for
conflict level and distance. For $\theta_0=0.2,\theta_1=1.5,t = 10.0$, the result can be show in Figure~\ref{fig:combSix}.

\begin{figure}[h]
\includegraphics[width=0.6\linewidth]{comb6.png}
\caption{Selected model}
\label{fig:combSix}
\end{figure}

The model seems perfect to our demand, following article will use it as baseline model.

\section{A framework and algorithms for inference}

A bayes inference framework is a programming language or a library that is made to organize the 
programming for model and inference related optmization code.

\subsection{Former inference framework}

There're a lot of inference framework such as Stan \cite{carpenter2017stan}, 
pymc\cite{patil2010pymc} and edward \cite{tran2016edward}. 
The core of them are a automated gradient computation library, 
they're stan-math, theano, tensorflow respectively. 

Nevertheless, those except stan are too weak in their ability describing complex probabiliy function.
And the compile time of stan is too long. 
So a light-weight bayes inference library is proposed to fulfill our goal. 
The framework is based on pytorch providing a flexible automated gradient computation lib and 
easy-to-use dynamic computation graph like Chainer.

\subsection{Inference method and algorithm introduction}

In this section, three inference method used below sections will be introduced.

\subsubsection{Maximum A Posteriori}

Maximum A Posteriori(MAP) is a estimator that maximize the likelihood function 
coporating prior information. Since the task of gradient computation is done by pytorch, 
we can only call the SGD or any other artful NN(neural net)-oriented optimizer to do standard gradient 
descent (take $loss = -likelihood \times prior$) process. See Algorithm~\ref{alg:sgd}.


\begin{algorithm}
\caption{Stochastic gradient descent}
\begin{algorithmic}[1]
\Procedure{SGD}{$\theta,lr,step$} \Comment{$\theta$ is init value, $lr$ is learning rate,$step$ is number of iterating}
    \For{i in 1:step}
        \State $\nabla \gets (\nabla_\theta \log(x,\theta)|_{\theta})$
        \State $\theta \gets \theta - lr \nabla -\theta$
    \EndFor
    \State \textbf{return} $\theta$
\EndProcedure
\end{algorithmic}
\label{alg:sgd}
\end{algorithm}

\subsubsection{Variational inference}

Variational inference use simple joint distribution setting to approximate complex 
exact posterior distribution, using Kullback-Leibler(KL) diversity as optimizing target \cite{blei2017variational}. That is:

$$
KL(q||p) = E_q \left( \log \frac{q(\theta \mid \mu,\omega)}{p(\theta \mid x)} \right)
$$

Where $p(\theta \mid x)$ is exact posterior distribution. 
$q(\theta)$ is corresponding variational distribution approximating the exact one.
The family of $q(\theta)$ are usually employed as normal distribution, especially $N(\mu,\mathbf{\sigma})$ form.
If $\mathbf{\sigma}=\mathrm{diag}(exp(\mathbf{\omega}))$, 
that is so called meanfield setting we will see it in following content.
%$q(\theta \mid \mu,\omega)$ is corresponding variational distribution approximating the exact one
%and the $\mu,\omega=\log(\sigma)$ means the distribution is in normal family with independent variable(meanfield).
%The $\mu,\omega$ will be omited in following content.

Since it employ variational(approximated) distribution $q(\theta)$ as expectation weight 
instead of exact posterior distribution,
the computation difficulty is relative reduced. But it seems too hard to deal too. 
So we turn to considering evidence lower bound(ELBO) which is lower bound for $\log p(x)$:

\begin{align*}
\log p(x) &= \log \int_\theta p(x,\theta) = \log \int_\theta p(x,\theta) \frac{q(\theta)}{q(\theta)} = \log \left( E_q \frac{p(x,\theta)}{q(\theta)} \right)  \\
          &\ge E_q (\log p(x,\theta)) - E_q(\log q(\theta)) = \mathrm{ELBO}
\end{align*}

The relation of KL diversity and ELBO can be show:

\begin{align*}
KL(q(\theta) || p(\theta \mid x)) &= E_q \log \frac{q(\theta)}{p(\theta \mid x)}  \\
                                  &= E_q \log q(\theta) - E_q \log p(\theta \mid x) \\
                                  &= E_q \log q(\theta) - E_q \log p(\theta,x) + E_q \log p(x) \\
                                  &= -(E_q \log p(\theta,x) -E_q \log q(\theta)) + \log p(x) \\
                                  &= -\mathrm{ELBO} + \log p(x)
\end{align*}

So we can maximizing ELBO to minimizing KL diversity. 
To do the optimization, the tradition variational inference method, 
Coordinate ascent variational inference (CAVI) algorithm,
require a analytic conditional expectation $E_{-j}(\log p(\theta_j \mid \mathbf{\theta}_{-j},\mathbf{x}))$. 
It's annoy when lazy statistician aware that he can do sampling to
reach same goal without any of such brain exhausted brain challenges. 

The Automatic differentiation variational inference (ADVI) is presented by 
\cite{kucukelbir2017automatic} and \cite{kucukelbir2014fully} to fix it. 
The core of the algorithm is to use random integral to approximate a hard-to-deal expectation.

Algorithm~\ref{alg:advi} is a simplified version from \cite{kucukelbir2017automatic}, benefited that we don't need transform
the support of restricted variable to unconstrained space.

\begin{algorithm}
\caption{Automatic differentiation variational inference(meanfield without transform)}
\begin{algorithmic}[1]
\Procedure{ADVI}{$\mathbf{\mu},\mathbf{\omega},lr,M,step$}  \Comment{$\mathbf{\omega}$ is init value, M is size of sample to monte carlo integral}
    \For{$s$ in $1:step$}
        \State $\hat{\nabla} \gets \mathbf{0}$
        \State $\mathbf{\eta} \sim N(\mathbf{0},\mathbf{I}) $ \Comment{Draw a sample from standard normal distribution}
        \For{$i$ in $1:M$} 
            \State $\hat{\theta} \gets (\mathbf{\eta}_i \exp(\mathbf{\omega}_i)) + \mathbf{\mu}_i$
            \State $\hat{\nabla} \gets \hat{\nabla} + (\nabla_\theta \log p(x,\theta)|_{\hat{\theta}})$
        \EndFor
        \State $\hat{\nabla} \gets \hat{\nabla} / M$
        \State $\mu \gets \mu + lr \hat{\nabla}$
        \State $\omega \gets \omega + lr \hat{\nabla} \mathbf{\eta}^T \mathrm{diag}({\exp(\omega)}) + \mathbf{1}$
    \EndFor 
    %\textbf{return} $\mathbf{\mu},\mathbf{\omega}$
    \State \Return $\mathbf{\mu},\mathbf{\omega}$
    %\State 
\EndProcedure
\end{algorithmic}
\label{alg:advi}
\end{algorithm}

The ADVI use noised gradient to find posterior expect that is equal to MAP(with more low effiecency),
 and collect gradient shock and concussion to estimate posterior variance.
 
In this paper, meanfield normal approximation is employed. That is:

$$
p(x^E_1,\dots,x^E_{N_E},y^E_1,\dots,y^E_{N_E}) \sim 
N((\mu_{X^E_1},\dots,\mu_{y^E_{N_E})^T},\mathrm{diag}(\sigma_{X^E_1},\dots,\sigma_{y^E_{N_E}}))
$$

Where $p(x^E_i) \sim N(\mu_{x^E_i},\sigma_{x^E_i}^2)$. 
Such $\mu_{x^E_i},\sigma_{x^E_i}^2)$ are called variational(approximated) parameters. 
it's called meanfield since the covariace matrix is a diagonal matrix, 
excluding all of correlation between randomized parameters.

\subsubsection{Sampling}

Variational inference is only a approximating method, 
sampling from posterior distribution directly is exact but resource expensive method. 
Classic Hasting-Metropolis method rely on a proposed distribution manually specified by statistician
and suffer bad converge effciency. 
The Hamiltonian Monte Carlo \cite{hoffman2014no} (See Algorithm~\ref{alg:hmc}) can be employed to improve the effciency,
using gradient information instead of proposed distribution to give next step in random walking. .

\begin{algorithm}
\caption{Mamitonian Monte Carlo}
\begin{algorithmic}[1]
\Procedure{HMC}{$\theta_0,\epsilon,L,step$} \Comment{$L$ and $\epsilon$ are leap-frog parameters}
    \For{$i$ in $1:step$}
        \State $r_0 \sim N(0,I)$
        \State $\theta_i \gets \tilde{\theta} \gets \theta_{i-1}$
        \State $\tilde{r} \gets r_0$
        \For{$l$ in $1:L$} \Comment{leap-frog process}
            \State $\tilde{r} \gets \tilde{r} + (\epsilon/2) \nabla_\theta P(X,\theta)|_{\tilde{\theta}}$
            \State $\tilde{\theta} \gets \tilde{\theta} + \epsilon \tilde{r}$
            \State $\tilde{r} \gets \tilde{r} + (\epsilon/2) \nabla_\theta P(X,\theta)|_{\tilde{\theta}}$
        \EndFor
        \State $\alpha \gets \min \left\{ 1, \frac{\exp(P(x,\tilde{\theta})-\frac{1}{2}\tilde{r}\cdot\tilde{r})}{\exp(P(x,\theta_{i-1})-\frac{1}{2}r_0\cdot r_0)} \right\}$
        \State $u \sim Uniform(0,1)$
        \If{$u \le \alpha$}
            \State $\theta^i \gets \tilde{\theta}$
        \EndIf
    \EndFor
    \State \Return $\theta$
\EndProcedure
\end{algorithmic}
\label{alg:hmc}
\end{algorithm}

HMC follow the same way as ADVI using gradient pulse and bounce.

HMC and other sampling method return a "trace" sample that can be used 
to build empirical posterior distribution. In this paper, the trace almost always been see as
parameters of normal distribution to compare to VB(ADVI) thougn it drop the main advantage of sampling method.

\section{Inference experiments}

\subsection{Enemy position detecting}

The situation (see Figure~\ref{fig:expState}) will be used as inference experiment data.

\begin{figure}[h]
\includegraphics[width=0.6\linewidth]{exp_state.png}
\caption{Given situation}
\label{fig:expState}
\end{figure}


\subsubsection{MAP}

Firstly, if flat prior is used (only provide likelihood), the MAP result is Figure~\ref{fig:MAPone}.

\begin{figure}[h]
\includegraphics[width=0.6\linewidth]{MAP1.png}
\caption{MAP with flat prior}
\label{fig:MAPone}
\end{figure}

The arrowtail indicate the "true" position of enemy(though it can't be seen), 
the arrowhead indicate the estimated position of enemy for MAP. As we can see,
a enemy is pulled to the back of ally units to maximize probability, it seems overfit anyway.
We can utilize the advatage of bayes framework, adding a prior \footnote{Or regulation term, but I
prefer the probability meaning offered by bayes.}:


$$
P_{\text{enemy}}(x,y) = \exp(sigm(t((x+y) - \alpha)))
$$

Where $t$ is tense parameter for enemy position,
$\alpha$ is corresponding threshold. We set $t=5.0,\alpha=5.0$
for illustration below.

The new joint probability with the diagonal prior is:

\begin{align*}
\log p(\mathbf{x}^E,\mathbf{y}^E) &= \sum_{i=1}^B \log sigm(t (P_\text{ally}(x^B_i,y^B_i)(1-P_\text{ally}(x^B_i,y^B_i)) - \theta_0)) \\
                                  &+ \sum_{i=1}^B \log sigm(t(-\min_{u} distance(x^B_i,y^B_i) + \theta_1)) \\
                                  &+ \sum_{j=1}^E sigm(t((x^E_j+y^E_j) - \alpha))
\end{align*}

Where $B,E$ are number of battle and enemy,$x^B_i,y^B_i$ is coordinate of battle $i$, $x^E_j,y^E_j$
is coordinate of enemy $j$.

The result of MAP on the new probability function looks better as shown in the Figure~\ref{fig:MAPtwo}.

\begin{figure}[h]
\includegraphics[width=0.6\linewidth]{MAP2.png}
\caption{MAP with diagonal prior}
\label{fig:MAPtwo}
\end{figure}

\subsubsection{Sampling and Variational inference}

Point estimation is not enough since it can't give uncertain information about the estimation.
Tough exact posterior may fix it but it's hard to compute and represent. Two ways in bayes inference 
to do it are sampling and variational inference(VI). The former one generate a "trace" which can be thought
as a sample taken from exact posterior, the trace may contitude empirical distribution or reveal 
attribution of the random parameter. The VI is faster but lack exact attribution.

The two results of VI in flat prior and new diagonal prior can be seen in Figure~\ref{fig:VI}.
The two results of sampling can be seen in Figure~\ref{fig:samping}.

\begin{figure}[h]
  \begin{subfigure}[b]{0.45\linewidth}
    \includegraphics[width=\linewidth]{VI11.png}
    \caption{VI with flat prior A}
  \end{subfigure}
  \begin{subfigure}[b]{0.45\linewidth}
    \includegraphics[width=\linewidth]{VI12.png}
    \caption{VI with diagonal prior A}
  \end{subfigure}
  \begin{subfigure}[b]{0.45\linewidth}
    \includegraphics[width=\linewidth]{VI21.png}
    \caption{VI with flat prior B}
  \end{subfigure}
  \begin{subfigure}[b]{0.45\linewidth}
    \includegraphics[width=\linewidth]{VI22.png}
    \caption{VI with diagonal prior B}
  \end{subfigure}
  \caption{VI fit result}
  \label{fig:VI}
\end{figure}

\begin{figure}[h]
  \begin{subfigure}[b]{0.45\linewidth}
    \includegraphics[width=\linewidth]{Sampling11.png}
    \caption{HMC with flat prior A}
  \end{subfigure}
  \begin{subfigure}[b]{0.45\linewidth}
    \includegraphics[width=\linewidth]{Sampling12.png}
    \caption{HMC with diagonal prior A}
  \end{subfigure}
  \begin{subfigure}[b]{0.45\linewidth}
    \includegraphics[width=\linewidth]{Sampling21.png}
    \caption{HMC with flat prior B}
  \end{subfigure}
  \begin{subfigure}[b]{0.45\linewidth}
    \includegraphics[width=\linewidth]{Sampling22.png}
    \caption{HMC with diagonal prior B}
  \end{subfigure}
  \caption{samping fit result(100 samples)}
  \label{fig:samping}
\end{figure}

Where the coordinate of ellipses indicate the expectation of those posterior distribution,
and the radius of ellipses indicate the standard deviation(sd) of axis x and y in every latent enemy point.
So those ellipse can be seen as a rough approximation for posterior distribution 
hidden in trace or variational distribution, though those two are not exact posterior too.

The HMC result seems inconsisitency, implying the converge fail. Though increasing sample size may be
solution(see Figure~\ref{fig:SamplingTen}). But just like we are not content to a NP-hard solver and claim that it only need more time,
sampling seems too weak in this problem.

\begin{figure}[h]
  \begin{subfigure}[b]{0.45\linewidth}
    \includegraphics[width=\linewidth]{Sampling31.png}
    \caption{HMC with flat prior A}
  \end{subfigure}
  \begin{subfigure}[b]{0.45\linewidth}
    \includegraphics[width=\linewidth]{Sampling32.png}
    \caption{HMC with diagonal prior A}
  \end{subfigure}
  \begin{subfigure}[b]{0.45\linewidth}
    \includegraphics[width=\linewidth]{Sampling41.png}
    \caption{HMC with flat prior B}
  \end{subfigure}
  \begin{subfigure}[b]{0.45\linewidth}
    \includegraphics[width=\linewidth]{Sampling42.png}
    \caption{HMC with diagonal prior B}
  \end{subfigure}
  \caption{samping fit result(1000 samples)}
  \label{fig:SamplingTen}
\end{figure}

\subsection{Mutiple enemy number setting}

The above estimation is based on "true" number of enemy. Let we test it's stability. 
What happen if the given number of number of enemy is wrong? The Figure~\ref{fig:bigVb} 
show the result, that flat prior seems well while diagonal prior may may be dominated by 
the prior. But it depict what we want in general.

\begin{figure}[h]
\includegraphics[width=0.99\linewidth]{big_vb.png}
\caption{Mutiple setting for approximation}
\label{fig:bigVb}
\end{figure}

\subsubsection{The density of probabity of existence of enemy estimation}

The one of advantage of VB is that it make computation of probability of specific region very easy.
The probability of exist of enemy in a given region $[x,x+dx]\times[y,y+dy]$ can be given by:

\begin{align*}
pe(x,y,dx,dy) = 1 - \prod_i (
& (1-(\Phi(x+dx - \mu_{X^E_i})/\sigma_{X^E_i}) -  (\Phi(x - \mu_{X^E_i})/\sigma_{X^E_i})) \\
& (1-(\Phi(y+dy - \mu_{Y^E_i})/\sigma_{Y^E_i}) -  (\Phi(y - \mu_{Y^E_i})/\sigma_{Y^E_i}))
)
\end{align*}

Where $\Phi(x)$ is standard cumulative probability function:

$$
\Phi(x) = \int_{-\infty}^x \frac{1}{\sqrt{2\pi}} \exp\left(\frac{x^2}{2}\right)
$$

And $$\mu^X_i$$ is the expect parameter of normal approximated posterior probability of ith enemy.
The three other are same.

The existence density(approximated) is defined:

$$
pe(x,y) = pe(x-\epsilon,y-\epsilon,2\epsilon,2\epsilon)/(4 \epsilon^2)
$$

Where $\epsilon$ is a small number, in there $\epsilon=0.01$. 
The two example can be seen in Figure~\ref{fig:existDensity}. 
The comparing result can be seen in seen in Figure~\ref{fig:bigVbExist}.

\begin{figure}[h]
\includegraphics[width=0.99\linewidth]{exist_density.png}
\caption{Two enlarged exist density situation}
\label{fig:existDensity}
\end{figure}


\begin{figure}[h]
\includegraphics[width=0.99\linewidth]{big_vb_exist_prob.png}
\caption{Mutiple setting fit with exist probabiliy}
\label{fig:bigVbExist}
\end{figure}



\section{Extention: Enemy movement detecting}

Considering ally,enemy,battles have not only positions, but also have timestamp. In another word,
enemy and ally may move in a time range, how do we infer the movement?

Interestingly, the model requires only a bit of modifications to integrate the new specification.

Assurming the time begin at $0$ and end at $1$, and enemy have uniform motion in constant speed.
So the parameters$x^E_i$ can be rewritten as $x^E_i(t) = (1-t)x^E_i(0) + tx^E_i(1)$, 
where $x^E_i(0),x^E_i(1)$ means the positions of ith enemy in time 0 and 1 respectively, and so on.
They are new parameters. The other structual probability function almost keep its original form.

First, assurming $x^E_i(0),y^E_i(0)$ are known and $x^E_i(1),y^E_i(1)$ are unknown.
A instance of inference is shown in Figure~\ref{fig:bkeu}. 
Where arrowtail indicate the known begin point.
The one of two arrowhead surrounded by sd ellipse is estimated $t=1$ coordination expectation. 
The another one is "true" target point.

\begin{figure}[h]
\includegraphics[width=0.4\linewidth]{bkeu.png}
\caption{The $t=0$ situation known but $t=1$ not}
\label{fig:bkeu}
\end{figure}

The following Figure~\ref{fig:bueu} show the situation in which $t=0,1$ situation are both unknown
\footnote{Note they're not in same data set.}. Where the arrowtail of first arrow is "true" begin
point, the middile point is estimated posterior expect of the point at $t=0$, the arrowhead of second arrow
is estimated posterior expect of the point at $t=1$.

\begin{figure}[h]
\includegraphics[width=0.6\linewidth]{bueu.png}
\caption{Both $t=0$ situation and $t=1$ are unknown}
\label{fig:bueu}
\end{figure}

The prior that give punishment to too long moving range also are useful and easy to implement. 
But the dataset that can highlight the advantage don't be constructed yet subject to time limit.
So it will be skiped.

\section{Extention: Dealing complex position distribution.}

The above contents are based on similar distribution, a curve can divide ally and enemy easily.
What will happen if the shape become complex and hard to capture for "one center" bayes naive classifer?

\subsection{Circle}

The "true" position is droped. This section will contain only "initialization value"
(In above content, "true position" play also initialization value role.). 
The Figure~\ref{fig:circleIteration} show the circle setting with flat prior, 
showing also a nice iteration convege process .


\begin{figure}[h]
\includegraphics[width=0.99\linewidth]{circle_iteration.png}
\caption{Iteration process for circle setting with flat prior and random initialization}
\label{fig:circleIteration}
\end{figure}

\subsection{Case study: The battle of Gettysburg}

Now, it's the time to abandon the globular chicken in vacuum loved by physicists.
The Figure~\ref{fig:gettysburg} show the map and corresponding data of the battle of Gettysburg in second day.
The data only capture the division(NATO:XX) and bridge brigade(NATO:X) in the map.
\footnote{You may aware the top right corner is not matched well. 
That is because The data depict state in morning and the map draw the situation in afternoon, the main battle time.
So it miss the march time the former one may infer.}
The confederate army have less number of divisions since the size of the divisions of CSA are more big.
The data is from HPS great wargame "Civil war:Gettysburg". 
The data used are listed in the Table~\ref{tab:Confederate} and \ref{tab:Union}, in which strength 
is number of unit that is not used in the model yet and will be used in later section as weight.

\begin{table}
\parbox{.45\linewidth}{
\begin{tabular}{lrrr}
\toprule
{} &  strength &    x &    y \\
\midrule
Heth's Div     &      4628 &  2.4 &  4.8 \\
McLaws' Div    &      6762 &  2.9 &  1.0 \\
Hood's Div     &      6957 &  3.3 &  0.1 \\
Anderson's Div &      6686 &  3.5 &  2.8 \\
Pender's Div   &      5080 &  3.6 &  3.7 \\
Rodes' Div     &      5202 &  5.0 &  4.2 \\
Early's Div    &      4572 &  6.0 &  3.9 \\
Johnson's Div  &      6012 &  7.2 &  4.5 \\
\bottomrule
\end{tabular}
\caption{Confederate oob}
\label{tab:Confederate}
}
\hfill
\parbox{.45\linewidth}{
\begin{tabular}{lrrr}
\toprule
{} &  strength &    x &    y \\
\midrule
2nd D (Humphreys)   &      4913 &  4.2 &  1.7 \\
1st Div (Birney)    &      5008 &  4.5 &  0.8 \\
2nd Div (Gibbon)    &      3558 &  5.1 &  2.3 \\
3rd Div (Hays)      &      3622 &  5.2 &  2.6 \\
1st Div (Caldwell)  &      3303 &  5.2 &  1.8 \\
3rd Div (Schurz)    &      1633 &  5.3 &  3.1 \\
2nd Div (Steinwehr) &      2264 &  5.4 &  2.9 \\
3rd Div (Doubleday) &      2922 &  5.4 &  2.7 \\
1st Div (Barlow)    &      1475 &  5.5 &  3.1 \\
2nd Div (Robinson)  &      1311 &  5.5 &  2.6 \\
1st Div (Wadsworth) &      1697 &  6.2 &  3.0 \\
2nd Div (Geary)     &      3851 &  6.2 &  2.8 \\
1st Div (Williams)  &      4698 &  6.3 &  2.1 \\
2nd Div (Ayres)     &      3990 &  6.6 &  1.6 \\
1st Div (Barnes)    &      3411 &  6.7 &  1.7 \\
3rd Div (Crawford)  &      2842 &  6.8 &  1.6 \\
3rd Div (Newton)    &      4729 &  7.3 &  0.8 \\
1st Div (Wright)    &      4181 &  7.4 &  1.2 \\
2nd Div (Howe)      &      3548 &  9.3 &  0.0 \\
\bottomrule
\end{tabular}
\caption{Union oob}
\label{tab:Union}
}
\end{table}

Following we assurm the position of army of Union and battle are known but CSA not.

\begin{figure}[h]
  \begin{subfigure}[b]{0.49\linewidth}
    \includegraphics[width=\linewidth]{gettysburg-map.png}
    \caption{Map of battle of Gettysburg, second day}
  \end{subfigure}
  \begin{subfigure}[b]{0.49\linewidth}
    \includegraphics[width=\linewidth]{gettysburg-model.png}
    \caption{Simple model for the situation}
  \end{subfigure}
  \caption{Comparing between real and model setting}
  \label{fig:gettysburg}
\end{figure}

In Figure~\ref{fig:gettysburgTwo} the forward probability model is illustrated and a sample of the model and observed
ally unit.

\begin{figure}[h]
  \begin{subfigure}[b]{0.49\linewidth}
    \includegraphics[width=\linewidth]{gettysburg-forward.png}
    \caption{Unnormalized probability of battle occuring}
  \end{subfigure}
  \begin{subfigure}[b]{0.49\linewidth}
    \includegraphics[width=\linewidth]{gettysburg-sample.png}
    \caption{A sample of occuring battle and ally unit.}
  \end{subfigure}
  \caption{The forward model and a sample of it}
  \label{fig:gettysburgTwo}
\end{figure}

We can run procedure described above with randomized initialized point. 
Figure~\ref{fig:gettysburgInit} show the exist-probability and converge process.

\begin{figure}[h]
\includegraphics[width=0.99\linewidth]{gettysburg-init.png}
\caption{Iteration process for randomized initialized setting with flat prior in Gettysburg sample}
\label{fig:gettysburgInit}
\end{figure}

It looks good, thougn the enemy units seems too close to ally. It can be solved a bit by prior.


\section{Extention: Size effect}

The above content assurms battles and units are homogeneous. It may make sense,
but if battle and unit can include number and size information, the result will be more useful.

To extend our model to embrace the change, we can merely define the high weight the unit have,
the long radius in distance factor and large weight in estimation on center of class unit cause.

Following content use the strength of Table~\ref{tab:Confederate} and \ref{tab:Union} as weight.
The $\mu,\sigma$ used by naive bayes classifer is computed by weithted version now.
And the distance will be adujusted to $dist_{ij} = dist_{ij} \frac{\beta}{stre_i}$, the $\beta$
is seted to $6000$. So all of units of Union will be penalized, 
and a few of Confederate will be bonused for distance factor.

The Figure~\ref{fig:gettysburgInitTwo} show the result, it very like the result fited without size effect. 
It turn out that the model is steady for suct effect.

\begin{figure}[h]
\includegraphics[width=0.99\linewidth]{gettysburg-init2.png}
\caption{Iteration process for randomized initialized setting with flat prior with Size effect}
\label{fig:gettysburgInitTwo}
\end{figure}


\section{Conclusion}

We present a standard bayes inference framework 
\footnote{In fact, the content of the paper have been implmented as examples of a open-source library
hosted in GitHub, see: \url{https://github.com/yiyuezhuo/bayes-torch} where yiyuezhuo is the nickname of
the writer Yueyi Zhuo} 
in a classes of prolbem in which 
two class of point(ally and enemy) will generate another class of points(battle), 
when one class is hidden(enemy).

The forward model can transformed easily to backward model to do inference due to our framework.
The inference include point detection, movement dectection with prior or adujusted prior.
They're be checked its steady and converge performance by true initialization and random initialization.
A case study about battle of Gettysburg summary the content covered by the paper. 
I wish the result can be producation environment and my game desigmning.

Desigmning the forward model, doing its inference and balancing effciency and correctness 
are interesting challenge, but useful.

\section{Core code implementing the model}

This code section contain only the component defining the baseline model and do inference. 
The full code about figure outputing, parameter tuning and model variants code can be checked in 
\footnote{\url{https://github.com/yiyuezhuo/Undergraduate-thesis/tree/master/notebook}}.

%Anyway, the section is added to deal with foolish paper Duplicate check system, so you would not 
%expect its completeness since actually the source code is giant.

The top code on baseline model:

\begin{python}

# model
friend = Data(friend_point)
battle = Data(battle_point)
enemy = Parameter(enemy_point) # set real value as init value, though maybe a randomed init is more proper

logPC = Data(_logPC)

conflict_threshold = 0.2
distance_threshold = 1.0
tense = 10.0
alpha = 5.0
prior_threshold = 5.0
prior_tense = 5.0

def target():
    friend_enemy = torch.cat((friend, enemy),0)
    distance = cdist(battle, friend_enemy).min(dim=1)[0]
    

    mu = Variable(torch.zeros(2,2)) 
    sd = Variable(torch.zeros(2,2))
    
    mu[0,:] = friend.mean(dim=0)
    mu[1,:] = enemy.mean(dim=0)
    sd[0,:] = friend.std(dim=0)
    sd[1,:] = enemy.std(dim=0)
    
    conflict = torch.exp(norm_naive_bayes_predict(battle, mu, sd, logPC)).prod(dim=1)
    p = soft_cut_ge(conflict,conflict_threshold, tense = tense) * soft_cut_le(distance, distance_threshold, tense = tense)
    
    target= torch.sum(torch.log(p))
    return target

def target2():
    target1 = target()
    # location prior
    target2 = target1 + torch.sum(enemy.sum(dim=1))
    return target2

\end{python}

The movement detecting model variant:

\begin{python}
def target():
    # 由于每个时点所涉及的mu,sd发生了变化所以不能像之前那样统一的处理,暂时也没看出显然的向量化写法
    target = Variable(torch.zeros(1))
    for i in range(battle_point.shape[0]):
        t = timestamp[i]
        friend = friend0*(1-t) + friend1*t
        enemy = enemy0*(1-t) + enemy1*t
        single_battle = torch.unsqueeze(battle[i],0) 
        
        friend_enemy = torch.cat((friend, enemy), 0)
        distance = cdist(single_battle, friend_enemy).min(dim=1)[0]
        
        mu = Variable(torch.zeros(2,2)) 
        sd = Variable(torch.zeros(2,2))

        mu[0,:] = friend.mean(dim=0)
        mu[1,:] = enemy.mean(dim=0)
        sd[0,:] = friend.std(dim=0)
        sd[1,:] = enemy.std(dim=0)

        conflict = torch.exp(norm_naive_bayes_predict(single_battle, mu, sd, logPC)).prod(dim=1)
        p = soft_cut_ge(conflict,conflict_threshold, tense = tense) * soft_cut_le(distance, distance_threshold, tense = tense)

        target+= torch.sum(torch.log(p))
    return target

\end{python}

The weighted model variant about forward and backward:

\begin{python}
# model for forward
reset()

friend = Data(friend_point)
battle = Data(xy)
enemy = Parameter(enemy_point) # set real value as init value, though maybe a randomed init is more proper

enemy_weight = Data(Confederate_weight)
friend_weight = Data(Union_weight)
enemy_dist_factor = Data(Confederate_dist_factor)
friend_dist_factor = Data(Union_dist_factor)


logPC = Data(_logPC)

conflict_threshold = 0.2
distance_threshold = 1.0
tense = 10.0
alpha = 5.0
prior_threshold = 5.0
prior_tense = 5.0

def target_p():
    friend_enemy = torch.cat((friend, enemy),0)
    friend_enemy_dist_factor = torch.cat((friend_dist_factor,enemy_dist_factor),0)
    
    distance = (cdist(battle, friend_enemy)*friend_enemy_dist_factor).min(dim=1)[0]
    #distance = distance * friend_enemy_dist_factor

    mu = Variable(torch.zeros(2,2)) #目前外层还有个同名的numpy.array mu,sd变量不要搞混了
    sd = Variable(torch.zeros(2,2))
    '''
    mu[0,:] = friend.mean(dim=0)
    mu[1,:] = enemy.mean(dim=0)
    sd[0,:] = friend.std(dim=0)
    sd[1,:] = enemy.std(dim=0)
    '''
    _friend_weight = torch.unsqueeze(friend_weight,1)
    _enemy_weight  = torch.unsqueeze(enemy_weight, 1)
    
    mu[0,:] = torch.sum(friend * _friend_weight,dim=0)
    mu[1,:] = torch.sum(enemy  * _enemy_weight, dim=0)
    sd[0,:] = torch.sqrt(torch.sum((friend - mu[0,:])**2 * _friend_weight, dim=0))
    sd[1,:] = torch.sqrt(torch.sum((enemy  - mu[1,:])**2 * _enemy_weight, dim=0))
    
    conflict = torch.exp(norm_naive_bayes_predict(battle, mu, sd, logPC)).prod(dim=1)
    p = soft_cut_ge(conflict,conflict_threshold, tense = tense) * soft_cut_le(distance, distance_threshold, tense = tense)
    return p

def target():
    p = target_p()
    
    target= torch.sum(torch.log(p))
    return target
    
# model for backward
reset()

friend = Data(friend_point)
battle = Data(battle_point)
enemy = Parameter(enemy_point) # set real value as init value, though maybe a randomed init is more proper

logPC = Data(_logPC)

conflict_threshold = 0.2
distance_threshold = 1.0
tense = 10.0
alpha = 5.0
prior_threshold = 5.0
prior_tense = 5.0


\end{python}

There're bayes-torch related code.

Class Model, defining the methods to do point estimation, vb and sampling.

\begin{python}

class Model:
    def __init__(self):
        self.parameters = []
        self.n_parameters = 0
        self.size_parameters = []
    def reset(self):
        self.parameters = []
        self.n_parameters = 0
        self.size_parameters = []
    def add_parameter(self, variable):
        self.parameters.append(variable)
        self.n_parameters += np.prod(variable.size())
        self.size_parameters.append(variable.size())
    def set_parameter_meanfield(self, mu, omega):
        n_parameters = self.n_parameters
        param_samples_eta = np.random.normal(size=n_parameters)
        param_samples = param_samples_eta*np.exp(omega) + mu
        self.set_parameter(param_samples)
        return param_samples_eta
    def set_parameter(self, values, is_float = True):
        # values is a flatten numpy array
        start = 0
        for param in self.parameters:
            param_size = np.prod(param.size())
            section = values[start:start+param_size].reshape(param.size())
            tensor = torch.from_numpy(section)
            if is_float:
                tensor = tensor.float()
            param.data = tensor
            start += param_size
    def collect_parameter_grad(self):
        grad = np.empty(self.n_parameters)
        start = 0
        for param in self.parameters:
            param_size = np.prod(param.size())
            grad[start:start+param_size] = param.grad.data.numpy().copy().ravel()
            start += param_size
        return grad
    def collect_parameter(self):
        res = np.empty(self.n_parameters)
        start = 0
        for param in self.parameters:
            param_size = np.prod(param.size())
            res[start:start+param_size] = param.data.numpy().copy().ravel()
            start += param_size
        return res
    def grad_q_meanfield(self, target_f, mu, omega, q_size=10, lr = 0.01):
        n_parameters = self.n_parameters
        
        mu_grad = np.zeros(n_parameters)
        omega_grad = np.zeros(n_parameters)
        
        optimizer = torch.optim.SGD(self.parameters, lr=lr)
        # optimizer only serve to zero_grad.
        
        for i in range(q_size):
            
            param_samples_eta = self.set_parameter_meanfield(mu, omega)
            
            optimizer.zero_grad()
            target = target_f()
            target.backward()
            
            param_grad = self.collect_parameter_grad()
            mu_grad += param_grad
            omega_grad += param_grad * param_samples_eta
        
        mu_grad /= q_size
        omega_grad /= q_size
        omega_grad *= np.exp(omega)
        omega_grad += 1.0
        
        return mu_grad,omega_grad
            
    def vb_meanfield(self, target_f, mu=None,omega=None, zero_init=False, 
                     n_epoch = 100, lr=0.01, q_size = 10):
        n_parameters = self.n_parameters
        
        if mu is None:
            if zero_init:
                mu = np.zeros(n_parameters)
            else:
                mu = np.array(self.collect_parameter())
        if omega is None:
            omega = np.zeros(n_parameters) # sigma=exp(omega) = 1
        
        #optimizer = torch.optim.SGD(self.parameters, lr=lr)
        
        for i in range(n_epoch):
            mu_grad,omega_grad = self.grad_q_meanfield(target_f, mu, omega, q_size = q_size)
            mu += lr * mu_grad
            omega += lr * omega_grad
        
        return mu,omega
        
        
    def vb_fullrank(self, target_f):
        raise NotImplementedError
    def vb_meanfield_format(self, res):
        # res = (mu, omega) -> {Parameter1: {'mu': mu1, 'sigma': exp(omega1)}...}
        mu, omega = res
        
        return_dict = {}
        start = 0
        for param in self.parameters:
            param_size = np.prod(param.size())
            _mu = mu[start:start+param_size].reshape(param.size())
            _omega = omega[start:start+param_size].reshape(param.size())
            return_dict[param] = {'mu':_mu,'omega':_omega}
            start += param_size
        return return_dict

    def vb(self, target_f, method = 'meanfield', format=False, 
                 reload=False, **kwargs):
        if method == 'meanfield':
            res = self.vb_meanfield(target_f, **kwargs)
            if not format:
                return res
            self.set_parameter(res[0])
            grad = self.grad(target_f) #当时的grad将被用来诊断是否收敛
            grad_size = np.sum(grad**2)
            converge = grad_size < 0.01
            report = dict( method = 'meanfiled', 
                           est = self.vb_meanfield_format(res),
                           grad = grad,
                           grad_size = grad_size,
                           converge = converge)
            if reload:
                self.set_parameter(res[0])
            return report
        if method == 'fullrank':
            return self.vb_fullrank(target_f, **kwargs)
        raise NotImplementedError
    def sampling_hmc(self, target_f, epsilon=0.01, L=20, M=100):
        
        def grad(theta):
            self.set_parameter(theta)
            return self.grad(target_f)
        def likelihood(theta):
            self.set_parameter(theta)
            return target_f().data.numpy()
        
        sample =[]
        sample.append(self.collect_parameter())
        accept_list = []
        
        for m in range(M):
            r0 = np.random.normal(size=self.n_parameters)
            r = r0
            theta0 = sample[-1]
            theta = theta0.copy()
            
            for i in range(L):
                r = r + 0.5 * epsilon * grad(theta)
                theta = theta + epsilon * r
                self.set_parameter(theta)
                r = r + 0.5 * epsilon * grad(theta)
            odd_up = np.exp(likelihood(theta)-0.5*np.dot(r,r))
            odd_bottom = np.exp(likelihood(theta0)-0.5*np.dot(r0,r0))
            alpha = min(1,odd_up/odd_bottom)
            accept_list.append(alpha)
            
            if np.random.random() < alpha:
                sample.append(theta)
            else:
                sample.append(theta0)
        return sample
        

    def sampling(self, target_f, method= 'hmc', **kwargs):
        #trace = []
        #raise NotImplementedError
        if method == 'hmc':
            return self.sampling_hmc(target_f, **kwargs)
        if method == 'nuts':
            return self.sampling_nuts(target_f, **kwargs)
        raise NotImplementedError
    def optimizing(self, target_f, lr=0.01, n_epoch = 1000):
        '''
        After optimizing run, the result is stored in original torch variable.
        target_f don't have any parameter, since the class serve to delete the trouble
        '''
        optimizer = torch.optim.SGD(self.parameters, lr=lr)
        
        for epoch in range(n_epoch):
            optimizer.zero_grad()
            target = target_f()
            loss = -target
            loss.backward()
            optimizer.step()
    def grad(self, target_f):
        fake_lr = 1.0
        optimizer = torch.optim.SGD(self.parameters, lr=fake_lr)
        optimizer.zero_grad()
        target = target_f()
        target.backward()
        return self.collect_parameter_grad()

\end{python}

Other helper function in bayes-torch:

\begin{python}
def torch_norm_log_prob(X,mu,sd):    
    temp = -(X - mu)**2/(2*sd) 
    temp = temp - 0.5 * torch.log(2.0 * np.pi * sd)
    return temp

def torch_norm_naive_bayes_predict(X,mu,sd,logPC):
    # X: sample_size * features
    # mu: class_size * features
    # sd: class_size * featrues
    # log_PC class_size
    n_class = logPC.size()[0]
    n_feature = X.size()[1]
    n_sample = X.size()[0]
    _X = torch_tile(X,[1,n_class]).resize(n_sample,n_class,n_feature)
    cp = torch_norm_log_prob(_X, mu, sd).sum(dim=2) + logPC  
    log_predict_prob = torch_transpose((torch_transpose(cp) - torch_logsumexp(cp,dim=1)))
    return log_predict_prob
    
def torch_logsumexp(X,dim=0):
    # numpy have this function
    return torch.log(torch.exp(X).sum(dim=dim))

def torch_tile(X,repeats):
    # torch.repeat ~ numpy.tile
    return X.repeat(*repeats)


def torch_soft_cut_ge(x,threshold,tense=1.0):
    return torch.sigmoid(tense*(x-threshold))

def torch_soft_cut_le(x,threshold,tense=1.0):
    return torch.sigmoid(tense*(-x+threshold))


\end{python}

Notebook specific code, the density estimation:

\begin{python}

def prob_point(x,y,dx,dy,_mu,_omega):
    mu = _mu.reshape(enemy_point.shape)
    sigma = np.exp(_omega.reshape(enemy_point.shape))
    x = np.reshape(x,(-1,1))
    y = np.reshape(y,(-1,1))
    px  = stats.norm.cdf((x-mu[:,0])/sigma[:,0])
    pdx = stats.norm.cdf((x+dx-mu[:,0])/sigma[:,0])
    py  = stats.norm.cdf((y-mu[:,1])/sigma[:,1])
    pdy = stats.norm.cdf((y+dy-mu[:,1])/sigma[:,1])
    return (pdx - px) * (pdy - py)

def prob_exist(x,y,dx,dy,_mu,_omega):
    ememy_prob = prob_point(x,y,dx,dy,_mu,_omega)
    return 1-np.prod((1 - ememy_prob).T,axis=0) 

def prob_exist_limit(x,y,_mu,_omega,step=0.1):
    p = prob_exist(x-step,y-step,2*step,2*step,_mu,_omega)
    return p/(4*step*step)

def show_change(show=False):
    display_data()
    for i in range(enemy_point.shape[0]):
        #s = 0.1
        plt.arrow(enemy_point[i][0], enemy_point[i][1], enemy.data[i][0] - enemy_point[i][0], enemy.data[i][1] - enemy_point[i][1],head_width=0.1)
    plt.legend()
    if show:
        plt.show()


def show_ellipse(mu,sd):
    from matplotlib.patches import Ellipse
    
    #ax = plt.subplot(111)
    ax = plt.gca() # get current axe, the lame method to support command style
    show_change(show=False)
    #res_reshaped = [r.reshape(enemy_point.shape) for r in res]
    for i in range(enemy_point.shape[0]):
        mu_x,mu_y = mu[i]
        sd_x,sd_y = sd[i]
        e=Ellipse((mu_x,mu_y), sd_x, sd_y, 0)
        e.set_clip_box(ax.bbox)
        e.set_alpha(0.1)
        ax.add_artist(e)
        
    #plt.show()
    
    
def show_vb(vb_res):
    res = vb_res
    model.set_parameter(res[0])
    res_reshaped = [r.reshape(enemy_point.shape) for r in res]
    mu = res_reshaped[0]
    sd = np.exp(res_reshaped[1])
    show_ellipse(mu,sd)
    
def multi_enemy_setting_show_exist_density(_enemy_point, target, title_string):
    reset_enemy(_enemy_point)
    res = vb(target)
    show_vb(res)
    p_exist = prob_exist_limit(xy[:,0],xy[:,1],res[0],res[1],step=0.001).reshape(xx.shape)
    CS = plt.contour(xx,yy,p_exist)
    #plt.contour(xx, yy, p_exist)
    plt.clabel(CS)
    #plt.title(title_string.format(i))
    plt.gca().set_title(title_string.format(i))
    plt.legend()

    
    
plt.figure(figsize=(16,20)) #(x,y)
for i in range(2, 10):
    interp_x = np.linspace(_enemy_point[:,0].min(),_enemy_point[:,0].max(),i)
    interp_y = np.interp(interp_x, _enemy_point[:,0], _enemy_point[:,1])
    interp_xy = np.c_[interp_x,interp_y]
    plt.subplot(4, 4, i-1)
    multi_enemy_setting_show_exist_density(interp_xy, target , '{} with flat prior')

for i in range(2, 10):
    interp_x = np.linspace(_enemy_point[:,0].min(),_enemy_point[:,0].max(),i)
    interp_y = np.interp(interp_x, _enemy_point[:,0], _enemy_point[:,1])
    interp_xy = np.c_[interp_x,interp_y]
    plt.subplot(4, 4, 8+i-1)
    multi_enemy_setting_show_exist_density(interp_xy, target2 , '{} with diag prior')

    #multi_enemy_setting_show(interp_xy, target2, '{} with diag prior')
plt.title('multiple number setting for vb with exist density')
plt.show()


\end{python}



%\begin{python}
%\end{python}

\bibliography{paper} 
\bibliographystyle{ieeetr}

\end{document}